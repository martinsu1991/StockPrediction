%%%%%%%%%%%%%%%%%%%%%%%%%%%%%%%%%%%%%%%%%
% Dreuw & Deselaer's Poster
% LaTeX Template
% Version 1.0 (11/04/13)
%
% Created by:
% Philippe Dreuw and Thomas Deselaers
% http://www-i6.informatik.rwth-aachen.de/~dreuw/latexbeamerposter.php
%
% This template has been downloaded from:
% http://www.LaTeXTemplates.com
%
% License:
% CC BY-NC-SA 3.0 (http://creativecommons.org/licenses/by-nc-sa/3.0/)
%
%%%%%%%%%%%%%%%%%%%%%%%%%%%%%%%%%%%%%%%%%

%----------------------------------------------------------------------------------------
%	PACKAGES AND OTHER DOCUMENT CONFIGURATIONS
%----------------------------------------------------------------------------------------

\documentclass[final,hyperref={pdfpagelabels=false}]{beamer}

\usepackage[orientation=portrait,size=a0,scale=1.4]{beamerposter} % Use the beamerposter package for laying out the poster with a portrait orientation and an a0 paper size

\usetheme{I6pd2} % Use the I6pd2 theme supplied with this template

\usepackage[english]{babel} % English language/hyphenation

\usepackage{amsmath,amsthm,amssymb,latexsym} % For including math equations, theorems, symbols, etc

%\usepackage{times}\usefonttheme{professionalfonts}  % Uncomment to use Times as the main font
%\usefonttheme[onlymath]{serif} % Uncomment to use a Serif font within math environments

\boldmath % Use bold for everything within the math environment

\usepackage{booktabs} % Top and bottom rules for tables

\graphicspath{{figures/}} % Location of the graphics files

\usecaptiontemplate{\small\structure{\insertcaptionname~\insertcaptionnumber: }\insertcaption} % A fix for figure numbering

%----------------------------------------------------------------------------------------
%	TITLE SECTION 
%----------------------------------------------------------------------------------------

\title{\huge  ~\vspace{10mm} \newline  Multi Data Source Stock Market Prediction} 


\author{\huge~\vspace{10mm} \newline  Yuhan Su, Ruichuang Cao, Wei Xu} % Author(s)


\institute{\huge Tsinghua University} % Institution(s)

%----------------------------------------------------------------------------------------
%	FOOTER TEXT
%----------------------------------------------------------------------------------------

\newcommand{\leftfoot}{Tsinghua University. http://iiis.tsinghua.edu.cn} % Left footer text

\newcommand{\rightfoot}{IBM SuperVessel Cloud. https://ptopenlab.com} % Right footer text

%----------------------------------------------------------------------------------------

\begin{document}
\linespread{0.98}	
\addtobeamertemplate{block end}{}{\vspace*{3ex}} % White space under blocks

\begin{frame}[t] % The whole poster is enclosed in one beamer frame

\begin{columns}[t] % The whole poster consists of two major columns, each of which can be subdivided further with another \begin{columns} block - the [t] argument aligns each column's content to the top

\begin{column}{.03\textwidth}\end{column} % Empty spacer column

\begin{column}{.445\textwidth} % The first column

%----------------------------------------------------------------------------------------
%	INTRODUCTION
%----------------------------------------------------------------------------------------

\begin{block}{Introduction}
\begin{itemize}	
\item For hundreds of years, everyone dreams to predict stock price changes. Numerous studies have shown that stock could be predicted to some degree. 
\item China stock market is influenced by rumors on social media.
\end{itemize}
\centering
\begin{figure}
	\includegraphics[width=0.6\linewidth]{intro.png}
	%\caption{Rumors in stock market}
	\label{intro}
\end{figure}
\begin{itemize}
 \item We design and implement a real-time data stream stock prediction system based on IBM SuperVessel Cloud.  
\item With the system, we can perform stock prediction based on multiple data sources, including trading data and social media rumors. 

\end{itemize}
\end{block}

%----------------------------------------------------------------------------------------
%	OBJECTIVES
%----------------------------------------------------------------------------------------

\begin{block}{Contribution}

\begin{itemize}
\item Build a system to process real-time data stream from multiple sources. 
\item Use multi data source, including trading data and social rumors, to predict stock market in China.
\item Leverage the state-of-the-art cloud technology to provide a scalable system. 
\item Provide an intuitive web UI to let users edit and analyze related information.
\end{itemize}
\centering
\begin{figure}
\includegraphics[width=0.5\linewidth]{intuition.png}
%\caption{Data for prediction}
\label{sample}
\end{figure}

\end{block}

%----------------------------------------------------------------------------------------
%	MATERIALS
%----------------------------------------------------------------------------------------

\begin{block}{Data Source}

%\begin{columns} % Subdivide the first main column

%\begin{column}{.56\textwidth} % The first subdivided column within the first main column

Currently we have the following datasets and we are adding more. 
\begin{itemize}
\item SSE50 index. We use its daily closing price. This is also our predicting target.
\item Sina Stock Forum. We use posts and comments from \href{http://guba.sina.com.cn}{Sina Guba}.

%\end{column}

%\begin{column}{.4\textwidth} % The second subdivided column within the first main column

%\end{column}
%\end{columns} % End of the subdivision

\item Financial news. These news are collected by Tushare website.
\item NASQAF index. We use its daily closing price.
\item RMB Exchange rate. We use its daily value.
\end{itemize}

\centering
\begin{figure}
\includegraphics[width=1.0\linewidth]{datasource.png}
%\caption{Multi data source}
\label{sample}
\end{figure}

\end{block}


\begin{block}{Data Pre-Processing}

\begin{itemize}
\item We perform multi-step data pre-processing on cloud-based streaming system. For data like rumors and news, we extract their sentiment as features, just like the left figure shows. For trading history data, we use their values within a certain window, as the right figure shows. 
\begin{columns} % Subdivide the first main column
\begin{column}{.4\textwidth} 
	\begin{figure}
		\includegraphics[width=1.2\linewidth]{textdata.png}
		%\caption{Rumors/News data}
		\label{text}
	\end{figure}
\end{column}
\begin{column}{.4\textwidth} 
	\begin{figure}
		\includegraphics[width=0.8\linewidth]{numdata.png}
		%\caption{Trading history data}
		\label{num}
	\end{figure}
\end{column}
\end{columns}

\end{itemize}
\end{block}


\end{column} % End of the first column

\begin{column}{.02\textwidth}\end{column} % Empty spacer column

\begin{column}{.445\textwidth} % The second column

%----------------------------------------------------------------------------------------
%	METHODS
%----------------------------------------------------------------------------------------

%----------------------------------------------------------------------------------------


%------------------------------------------------

\begin{block}{System Architecture}
	\begin{figure}
		\includegraphics[width=0.7\linewidth]{system.png}
		%\caption{System Architecture}
	\end{figure}
\begin{itemize}
\item Components in our system include Spark, HDFS, Gitlab, Django. They all run efficiently on SuperVessel Cloud. 
\item Our system integrates the acquisition, preprocessing and learning of data source streams, and shows the final prediction result. Users can monitor the whole process all from a single web UI. They can change the model parameters, modify the codes, and check the running status online.

\end{itemize}

\begin{figure}
		\includegraphics[width=0.5\linewidth]{webui.png}
		%\caption{Web UI}
	\end{figure}

\end{block}

\begin{block}{SuperVessel Cloud}
\begin{columns} % Subdivide the first main column
\begin{column}{.3\textwidth} 
	\begin{figure}
		\includegraphics[width=0.7\linewidth]{ibm.png}
		%\caption{SuperVessel Cloud}
		\label{sv}
	\end{figure}
\end{column}
\begin{column}{.3\textwidth} 
	\begin{figure}
		\includegraphics[width=1.0\linewidth]{svcloud.png}
		%\caption{SuperVessel Cloud}
		\label{sv}
	\end{figure}
\end{column}
\begin{column}{.3\textwidth} 
	\begin{figure}
		\includegraphics[width=0.7\linewidth]{svlogo.png}
		%\caption{SuperVessel Cloud}
		\label{svl}
	\end{figure}
\end{column}
\end{columns}

\begin{itemize}
\item This Project is developed on SuperVessel Cloud, which  is based on OpenPOWER technology and provides the high efficiency cognitive computing infrastructure for frontier science with high performance heterogenous platform (GPU/FPGA).
\item SuperVessel consists of three parts: basic cognitive cloud service, cognitive computing service platform, and application acceleration store for new technology sharing. 
\item SuperVessel provides us with a scalable and easy-to-maintain cloud infrastructure to build our systems on. 
\end{itemize}



 \begin{figure}
		\includegraphics[width=0.7\linewidth]{openpower.png}
		%\caption{System Architecture}
	\end{figure}
\end{block}

%----------------------------------------------------------------------------------------
%	CONCLUSION
%----------------------------------------------------------------------------------------

\begin{block}{Conclusion}

\begin{itemize}
\item Using multiple data sources improves stock prediction. 
\item Cloud technology provides us with high efficiency cognitive computing infrastructure, makes it simple for users to analyze in an intuitive web UI.
\end{itemize}

\end{block}

%----------------------------------------------------------------------------------------
%	ACKNOWLEDGEMENTS
%----------------------------------------------------------------------------------------

\begin{block}{Acknowledgments}

\begin{itemize}
\item Thank IBM China Research Lab for providing computing resources on SuperVessel Cloud. 
\item Thank Tushare website for providing datasets. 
\end{itemize}

\end{block}

%----------------------------------------------------------------------------------------
%	CONTACT INFORMATION
%----------------------------------------------------------------------------------------

\setbeamercolor{block title}{fg=black,bg=orange!70} % Change the block title color

\begin{block}{Contact Information}

\begin{itemize}
\item Web: http://iiis.tsinghua.edu.cn ~~ https://ptopenlab.com \\


\item Email: Yuhan Su~~~~~~~~~~\href{mailto:syhmartin@yeah.ne}{syhmartin@yeah.net} \\ 
~~~~~~~~~~Ruichuang Cao~~\href{mailto:create0818@163.com}{create0818@163.com} \\
~~~~~~~~~~Prof. Wei Xu~~~~~~\href{mailto:wei.xu.0@gmail.com}{wei.xu.0@gmail.com}
\end{itemize}

\end{block}

%----------------------------------------------------------------------------------------

\end{column} % End of the second column

\begin{column}{.015\textwidth}\end{column} % Empty spacer column

\end{columns} % End of all the columns in the poster

\end{frame} % End of the enclosing frame
\end{document}